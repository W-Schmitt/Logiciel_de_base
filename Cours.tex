\documentclass{article}
\usepackage[utf8]{inputenc}
\usepackage[french]{babel}
\usepackage[T1]{fontenc}
\usepackage{enumitem}
\usepackage{minted}
\usepackage{graphicx}
\usepackage{geometry}
\usepackage{caption}
\usepackage{float}
\usepackage{subcaption}
\usepackage{amsmath}
\usepackage{url}
\usepackage{dirtree}
\setlength{\parindent}{0pt}
\geometry{total={210mm,297mm},
left=25mm,right=25mm,
top=25mm,bottom=25mm}


\title{Logiciels de base}
\author{William SCHMITT}


\begin{document}

\maketitle

\section{Subtilités de C99 et rappels de ANSI C}

\subsection{Printf et Scanf}

Utiliser PRId32 ou SCNd32 pour les print/scan remplace ces valeurs par les bons
caractères de formatage.

\subsection{Tableaux}

\paragraph{Tableaux à taille variable}
\begin{minted}{C}
  uint16_t taille;
  scanf("%" SCNu16, &taille);
  int32_t tab[taille];
\end{minted}

\paragraph{Tableaux 2D}
\begin{minted}{C}
  /**
   * Les tableaux 2D sont à plat en mémoire.
   */
  char tab[3][5];
\end{minted}

\end{document}